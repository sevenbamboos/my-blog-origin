\documentclass{beamer}

\title{BEYOND TRY-CATCH}
\subtitle{Exception Handling in Java}
\author{\texttt{sam.wang@agfa.com}} 
\institute{Shanghai}
\date{\today}

%\usetheme{Bergen}

\begin{document}

\begin{frame}
	\titlepage
\end{frame}

\begin{frame}
	\frametitle{Outline}
	\tableofcontents[pausesections]
\end{frame}

\section{Plain Old Try}
\subsection{Checked Exception}
\begin{frame}
  \frametitle{Checked Exception}
  \begin{definition}
    Checked at compile time. If the code in a method throws a checked exception, then the method must either \alert{1.handle the exception} or it must specify the exception \alert{2.using throws} keyword.
  \end{definition}
  \begin{example}
    \begin{itemize}
    \item java.io.FileNotFoundException
    \item java.lang.InterruptedException
    \item java.sql.SQLException
    \end{itemize}
  \end{example}
\begin{uncoverenv}<2>
Not a best practice
\\Not implemented by modern languages later than Java
\end{uncoverenv}
\end{frame}

\subsection{Unchecked Exception}
\begin{frame}
  \frametitle{Unchecked Exception}
  \begin{definition}
    Not checked at compiled time. Exceptions under \alert{Error} and \alert{RuntimeException} classes are unchecked exceptions, everything else under throwable is checked.
  \end{definition}
  \begin{example}
    \begin{itemize}
    \item java.lang.NullPointerException
    \item java.lang.ArrayIndexOutOfBoundsException
    \item java.lang.OutOfMemoryError
    \end{itemize}
  \end{example}
\end{frame}

\subsection{Try-Catch Sample}
\begin{frame}[fragile]
  \frametitle{Try-Catch Sample}
PatientList find(String strAge, String strCreated)
\scriptsize{
\begin{verbatim}
// Process input age
int age = -1;
try {
  age = Integer.parseInt(strAge);
} catch (NumberFormatException e) {
  throw new IllegalArgument("Invalid age...}

// Process input date
Date created = null;
try {
  created = dateFormat.parse(strCreated);
} catch (ParseException e) {
  throw new IllegalArgument("Invalid create date...);
}

// Major logic starts here
age = age & 0x7F;
created = created.later(today) ? today : created;
patientService.search(age, created);
\end{verbatim}
} 
\end{frame}

\section{Functional Programming}
\subsection{Algebraic Data Type (ADT)}
\begin{frame}
  \frametitle{Algebraic Data Type (ADT)}
  \begin{block}{$\lambda$ Calculus}<1->
    \begin{enumerate}
    \item $Void \rightarrow 0$, $Unit, () \rightarrow 1$, $f(\alpha) \rightarrow \beta \rightarrow {\beta_{size}}^{\alpha_{size}}$
    \item $A | B \rightarrow A_{size} + B_{size}$
    \\ WorkDay $|$ Weekend $\rightarrow$ Week (5 + 2 = 7)
    \item $(A, B) \rightarrow A_{size} \cdot B_{size}$
    \\ $Tuple\left [WorkDay, Weather\right]$ $\rightarrow$ $WorkDay_{days} \cdot Weather_{kinds}$
    \end{enumerate}
  \end{block}
  \begin{example}<2->
  \alert{LinkedList:}$Nil | \Big(a, \big(List({a}')\big)\Big) \rightarrow \theta(a) = 1 + a \cdot \theta({a}')$ 
  $\frac{1}{1-x} \rightarrow 1+x+x^2+x^3... \rightarrow$ a list is either empty or containing a single element, or two elements, or three ...
  \alert{BinaryTree:}$Nil | Node\big(a, Tree(a), Tree(a)\big) \rightarrow \theta(a) = 1 + a \cdot \theta({a}')^2$ 
  $\frac{1-\sqrt{1-4x}}{2x} \rightarrow 1+a+2\cdot a^2 + 5 \cdot a^3 ... \rightarrow$ a binary tree is either empty or containing a value of type a, or two values of type a in two ways, or three values of type a in five different ways ...
  \end{example} 
\end{frame}

\subsection{Design Pattern in Functional Programming}
\begin{frame}
  \frametitle{Design Pattern in Functional Programming}
  \begin{block}{Functor}<1->
  Apply a function to a wrapped value. m a $\rightarrow$ (a $\rightarrow$ b) $\rightarrow$ m b
  \\ $Optional \left [T \right]::map(f: T \rightarrow U): Optional \left [U \right ]$
  \\ $Stream \left [T \right]::map(f: T \rightarrow U): Stream \left [U \right ]$
  \end{block}
  \begin{block}{Applicatives}<2->
  Apply a wrapped function to a wrapped value
  \\ \textit{m a $\left <* \right >$ n b $\rightarrow$ Just (a.apply b)}
  \end{block} 
  \begin{block}{Monad}<3->
  Apply a function that returns a wrapped value, to a wrapped value. m a $\rightarrow$ (a $\rightarrow$ m b) $\rightarrow$ m b
  \\ $Optional \left [T \right ] ::flatMap(f: T \rightarrow Optional \left [U \right ]): Optional \left [U \right ]$
  \\ $Stream \left [T \right ]::flatMap(f: T \rightarrow Stream \left [U \right ]): Stream \left [U \right ]$
  \end{block}
\end{frame}

\subsection{Better Way in Java 8}
\begin{frame}[fragile]
  \frametitle{Better Way in Java 8}
  \begin{block}{java.util.Optional}
  $\verb|tryParseInt(s: String): Optional<Integer>| \in Integer$
  $\verb|tryParse(s: String): Optional<Date>| \in DateFormat$
  \verb|find(String age, String created): Optional<PatientList>|
  \end{block}
  \begin{block}{Solution}
  Expression-oriented: one compact expression, no temporary variables and 
  \alert{composability}\footnote{\scriptsize{
  $\beta=f(\alpha) , \gamma=g(\beta) \rightarrow \gamma=g\big(f(\alpha)\big)$
  }}
  \end{block}
  \scriptsize{
  \begin{verbatim}
  Integer.tryParseInt(age).flatMap(a ->
    dateFormat.tryParse(created).map(c -> {
      // Only when both age and created have no problem
      a = a & 0x7F;
      c = c.later(today) ? today : c;
      return patientService.search(a, c); 
    })
  );
  \end{verbatim}
  }
\end{frame}

\subsection{Problems with Optional}
\begin{frame}
  \frametitle{Problems with Optional}
    \begin{enumerate}
    \item<1-> Not adopted by many existing APIs $yet$.
      \begin{description}
      \item[-] $Map::get(key: String): Object \rightarrow null $
      \item[-] $String::indexOf(ch: int): int \rightarrow -1$
      \end{description}
    \item<2-> Happy path only. Nowhere to get exception details
    \item<3-> Ugly when too many nested \alert{$\lambda$}
        \begin{description}
        \item[Example] $opt1.flatMap(v1 \rightarrow opt2.flatMap(v2 \rightarrow opt3.flatMap(v3 \rightarrow opt4.map(v4 \rightarrow v1 + v2 + v3 + v4))));$
        \end{description}
    \item<4-> Not able to get all validation results.
    \end{enumerate}
\end{frame}
  
\section{Monad Try in Action}
\subsection{Try}
\begin{frame}[fragile]
  \frametitle{Try}
  interface $Try \left[R \right]$
  \scriptsize{
  \begin{verbatim}
<T> Try<T> map(Function<R,T> f);
<T> Try<T> flatMap(Function<R, Try<T>> f);

// Resolve with no worry about error
void forEach(Consumer<R> callback);
// Resolve with a success callback and an error handling
void andThen(Consumer<R> callback, Consumer<Throwable> errorHandling);
// Resolve separately by two steps
Try<R> ifSuccess(Consumer<R> callback);
void orElse(Consumer<Throwable> errorHandling);
// Miscellaneous methods
Try<R> filter(Predicate<R> f);
Optional<R> toOption();

static <T> Try<T> tryWith(Block<T> s) {
  try { return new Success(s.execute());
  } catch (Throwable e) {
    if (isFatal(e)) throw new RuntimeException(e);
    else return new Failure(e);
  }
}
  \end{verbatim}
  }
\end{frame}

\subsection{Failure}
\begin{frame}[fragile]
  \frametitle{Failure}
  final class Failure implements Try
  \scriptsize{
  \begin{verbatim}
private Throwable exception;
Failure(Throwable exception) { this.exception = exception; }

@Override public Try map(Function f) { return this; }
@Override public Try flatMap(Function f) { return this; }

@Override public void forEach(Consumer callback) { return; }
@Override public void andThen(Consumer callback, Consumer errorHandling) {
  orElse(errorHandling);
}
@Override public Try ifSuccess(Consumer callback) { return this; }
@Override public void orElse(Consumer errorHandling) {
  errorHandling.accept(exception);
}

@Override public Try filter(Predicate f) { return this; }
@Override public Optional toOption() { return Optional.empty(); }   
  \end{verbatim}
  }
\end{frame}

\subsection{Success}
\begin{frame}[fragile]
  \frametitle{Success}
  final class Success$\left<R \right>$ implements Try$\left<R \right>$
  \scriptsize{
  \begin{verbatim}
private R result;
Success(R result) { this.result = result; }
@Override public <T> Try<T> map(Function<R, T> f) {
  return Try.tryWith(() -> f.apply(result));
}
@Override public <T> Try<T> flatMap(Function<R, Try<T>> f) {
  Try<Try<T>> mapped = Try.tryWith(() -> f.apply(result));
  if (mapped.isSuccessful()) {
    return mapped.get();
  } else {
    return new Failure(mapped.exception());
  }
}
@Override public void forEach(Consumer<R> callback) {
  Try.tryWith(() -> { callback.accept(result); return null; });
}
@Override public Try<R> ifSuccess(Consumer<R> callback) {
  return Try.tryWith(() -> { callback.accept(result); return result; });
}
@Override public void andThen(Consumer<R> callback, 
    Consumer<Throwable> errorHandling) {
  ifSuccess(callback);
}
  \end{verbatim}
  }
\end{frame}

\subsection{Try Application}
\begin{frame}[fragile]
  \frametitle{Try Application}
  \scriptsize{
  \begin{verbatim}
Try<Integer> readAge = tryWith(Integer.parseInt(age));
Try<Date> readDate = tryWith(dateFormat.parse(created));

BiFunction<Integer,Date,PatientList> search = (a,c) -> {
  a = a & 0x7F;
  c = c.later(today) ? today : c;
  return patientService.search(a,c);
}

Try<PatientList> result = readAge.flatMap(a ->
  readDate.map(c -> search.apply(a, c)));

result.forEach(patientList -> ... );

result.ifSuccess(patientList -> ... ).orElse(exception -> ... );

result.andThen(
  patientList -> ... ,
  exception -> ... 
);
  \end{verbatim}
  }
\end{frame}

\subsection{Improve Try}
\begin{frame}[fragile]
  \frametitle{Improve Try}
    \begin{enumerate}
    \item Still ugly when too many $Try$s
    \item No lazy evaluation (No need to continue readDate when readAge failed)
    \item It would be nice to have:
    \end{enumerate}
  \scriptsize{
  \begin{verbatim}
for {a <- Integer.parseInt(age)
     c <- dateFormat.parse(created)} 
  yield(patientService.search(
    a & 0x7F, 
    c.later(today) ? today : c)
  )
  \end{verbatim}
  }
\end{frame}

\subsection{TryBuilder}
\begin{frame}[fragile]
  \frametitle{TryBuilder}
  final class TryBuilderN$\left <R1,R2,...RN \right>$
  \scriptsize{
  \begin{verbatim}
private Block<R1> b1;
private Block<R2> b2;
...
private Block<RN> bn;

static TryBuilderN forN(Block<R1> b1, Block<R2> b2, ... Block<RN> bn) {
  this.b1 = b1;
  this.b2 = b2;
  ...
  this.bn = bn;
}

public <T> Try<T> yield(MultiFunction<R1, R2, ...RN, T> f) {
  return
    Try.tryWith(b1).flatMap(v1 ->
      Try.tryWith(b2).flatMap(v2 ->
        ...
          Try.tryWith(bn).map(vn ->
            f.apply(v1, v2, ...vn))));
}
  \end{verbatim}
  }
\end{frame}

\subsection{TryBuilder Application}
\begin{frame}[fragile]
  \frametitle{TryBuilder Application}
  \scriptsize{
  \begin{verbatim} 
for2(() -> Integer.parseInt(age), 
     () -> dateFormat.parse(created))
     
  .yield((a,c) -> {
      a = a & 0x7F;
      c = c.later(today) ? today : c;
      return patientService.search(a,c);})
  
  .andThen(
      patientList -> ... ,
      exception -> ...
  );
  \end{verbatim}
  }
  \begin{block}{Next Step}
  \alert{Monad Validation} to support exception accumulation and return them once for all
  \end{block}
\end{frame}

\end{document}
