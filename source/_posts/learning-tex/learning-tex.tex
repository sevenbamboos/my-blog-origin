\documentclass{article}
\title{Learning Tex}
\author{Sam Wang}
\date{Oct 17, 2017}			

\usepackage{amsmath}

\newcommand{\site}[2][protocol]{\texttt{#1://#2}}
\newcommand{\fullref}[1]{\ref{#1} on page~\pageref{#1}}

\begin{document}

\maketitle

\tableofcontents

\section{Installation}
About Tex Distribution

On Windows, MiKTex is a good choice with not too large size. Note MiKTex has its own package manager (GUI of course, but not as same as that of TexLive) and make sure to use the best repository to fast-download packages. 

On Mac, I tried a basic installation package of MacTex (the full package is over 3G). Note MacTex uses a basic version of TexLive as package manager. So make sure Tex commands are in user path and have correct user/group and permission to get executed. Also choose the right repository with 
\\tlmgr option repository \site[http]{<mirror-domain>/systems/texlive/tlnet/} 

About Editor
MiKTex includes a front end, Texworks\label{texworks} while the basic MacTex does not. A better choice (for my taste) is \emph{Visual Code + Latex Workshop(plugin)}. Latex Workshop uses \emph{latexmk} to compile tex files by default. So make sure it is installed (via package manager). 

About Q\&A \cite{R02}

\section{Concept}
\begin{description}
\item[Preamble]Part before \verb|\begin{document}|
\item[Command]Start from $\backslash$
\item[Environment]Block to have settings in its own scope
\item[Font]\emph{serif} has small details at the end of letter strokes and thus is suitable for body
\\ \emph{sans-serif} for headings and text on low-resolution screens
\\ \texttt{monospace}(\verb|\texttt|) for source code. 
\item[Online document (in case no \emph{texdoc})]\site[http]{ctan.org/pkg/<package-name>} to find document
\end{description}

\section{Basic Style}
\subsection{Semantic markup}

\verb|\emph| is more flexible than \verb|\textit| in that it represents the \emph{\emph{intention} instead of a detailed style}.
Also note \emph can be nested and the effect is to flip between normal and italic.

\subsection{New Command}
Define new command inside preamble (before \verb|\begin{document}|):
\\ \verb|\newcommand{command-name}[parameter-count][optional-parameters]{definition}|
\verb|\site[git]{github.com/sevenbamboos}| $\rightarrow$ \site[git]{github.com/sevenbamboos}

\subsection{Item}
% Numbered list
\begin{enumerate}
  \item First action
  \begin{enumerate}
    \item Step one
    \item Step two
  \end{enumerate}
  \item Second action
  % Non-numbered list
  \begin{itemize}
    \item Member A
    % Description items
    \begin{description}
      \item[Courage] Average **
      \item[Intelligence] Poor *
      \item[Knowledge] Good ***
    \end{description}
    \item Member B
  \end{itemize}
\end{enumerate}

\subsection{Table}
\begin{tabular}{ccc}
  Name & User-friendly & Powerful\\
  \hline
  TeXworks (see \fullref{texworks}) & $\star\star$ & $\star\star\star$\\
  Visual Code + Latex Workshop & $\star\star\star\star$ & $\star\star$\\
\end{tabular}

\subsection{Image}
\subsection{Reference}
\verb|\label{key} \ref{key} \page{ref}|

\subsection{Math}
Check math symbols at \cite{R01}
\\Inline equation:
\( x_{1,2} = \frac{-b \pm \sqrt{b^2-4ac}}{2a} \)
\\Display equation:
\[ x_{1,2} = \frac{-b \pm \sqrt{b^2-4ac}}{2a} \]
\\Numbered equation:
\begin{equation}
x_{1,2} = \frac{-b \pm \sqrt{b^2-4ac}}{2a}
\end{equation}
\begin{align*}
  a + b + c &= 1\\
  b + c &= 0\\
  c &= 2
\end{align*}

\subsection{Source code}

\begin{thebibliography}{widest label}
  \bibitem[symbols]{R01} \site[https]{en.wikibooks.org/wiki/LaTeX/Mathematics}
  \bibitem[Q\&A]{R02} \site[https]{tex.stackexchange.com}
\end{thebibliography}

\end{document}