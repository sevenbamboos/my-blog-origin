\documentclass{article}
\title{Learning Tex}
\author{Sam Wang}
\date{Oct 17, 2017}			

\newcommand{\site}[2][protocol]{\texttt{#1://#2}}

\begin{document}

\maketitle

\tableofcontents

\section{Concept}
\begin{description}
\item[Preamble]Part before \verb|\begin{document}|
\item[Command]Start from $\backslash$
\item[Environment]Block to have settings in its own scope
\item[Font]\emph{serif} has small details at the end of letter strokes and thus is suitable for body
\\ \emph{sans-serif} for headings and text on low-resolution screens
\\ \texttt{monospace}(\verb|\texttt|) for source code. 
\item[Online document (in case no \emph{texdoc})]\site[http]{ctan.org/pkg/<package-name>} to find document
\end{description}

\section{Basic Style}
\subsection{Semantic markup}

\verb|\emph| is more flexible than \verb|\textit| in that it represents the \emph{\emph{intention} instead of a detailed style}.
Also note \emph can be nested and the effect is to flip between normal and italic.

\subsection{New Command}
Define new command inside preamble (before \verb|\begin{document}|):
\\ \verb|\newcommand{command-name}[parameter-count][optional-parameters]{definition}|
\verb|\site[git]{github.com/sevenbamboos}| $\rightarrow$ \site[git]{github.com/sevenbamboos}

\subsection{Item}
% Numbered list
\begin{enumerate}
  \item First action
  \begin{enumerate}
    \item Step one
    \item Step two
  \end{enumerate}
  \item Second action
  % Non-numbered list
  \begin{itemize}
    \item Member A
    % Description items
    \begin{description}
      \item[Courage] Average **
      \item[Intelligence] Poor *
      \item[Knowledge] Good ***
    \end{description}
    \item Member B
  \end{itemize}
\end{enumerate}

\subsection{Table}
\subsection{Image}
\subsection{Reference}
\verb|\label[name]{key}|

\subsection{Source code}
\end{document}